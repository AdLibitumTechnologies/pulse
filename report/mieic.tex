%% FEUP THESIS STYLE for LaTeX2e
%% how to use feupteses (English version)
%%
%% FEUP, JCL & JCF, 31 July 2012
%%
%% PLEASE send improvements to jlopes at fe.up.pt and to jcf at fe.up.pt
%%

%%========================================
%% Commands: pdflatex tese
%%           bibtex tese
%%           makeindex tese (only if creating an index)
%%           pdflatex tese
%% Alternative:
%%          latexmk -pdf tese.tex
%%========================================

\documentclass[11pt,a4paper,twoside,openright]{report}

%% For iso-8859-1 (latin1), comment next line and uncomment the second line
\usepackage[utf8]{inputenc}
%\usepackage[latin1]{inputenc}

%% English version

%% MIEIC options
\usepackage[mieic]{feupteses}
%\usepackage[mieic,juri]{feupteses}
%\usepackage[mieic,final]{feupteses}
%\usepackage[mieic,final,onpaper]{feupteses}

%% Additional options for feupteses.sty:
%% - onpaper: links are not shown (for paper versions)
%% - backrefs: include back references from bibliography to citation place

%% Uncomment the next lines if side by side graphics used
%\usepackage[lofdepth,lotdepth]{subfig}
%\usepackage{graphicx}
%\usepackage{float}

%% Include color package
\usepackage{color}
\definecolor{cloudwhite}{cmyk}{0,0,0,0.025}

%% Include source-code listings package
\usepackage{listings}
\lstset{ %
 language=C,                        % choose the language of the code
 basicstyle=\footnotesize\ttfamily,
 keywordstyle=\bfseries,
 numbers=left,                      % where to put the line-numbers
 numberstyle=\scriptsize\texttt,    % the size of the fonts that are used for the line-numbers
 stepnumber=1,                      % the step between two line-numbers. If it's 1 each line will be numbered
 numbersep=8pt,                     % how far the line-numbers are from the code
 frame=tb,
 float=htb,
 aboveskip=8mm,
 belowskip=4mm,
 backgroundcolor=\color{cloudwhite},
 showspaces=false,                  % show spaces adding particular underscores
 showstringspaces=false,            % underline spaces within strings
 showtabs=false,                    % show tabs within strings adding particular underscores
 tabsize=2,	                    % sets default tabsize to 2 spaces
 captionpos=b,                      % sets the caption-position to bottom
 breaklines=true,                   % sets automatic line breaking
 breakatwhitespace=false,           % sets if automatic breaks should only happen at whitespace
 escapeinside={\%*}{*)},            % if you want to add a comment within your code
 morekeywords={*,var,template,new}  % if you want to add more keywords to the set
}

%% Uncomment to create an index (at the end of the document)
%\makeindex

%% Path to the figures directory
%% TIP: use folder ``figures'' to keep all your figures
\graphicspath{{figures/}}

%%----------------------------------------
%% TIP: if you want to define more macros, use an external file to keep them
% text shorthands
%
\newcommand{\Feup}{Faculdade de Engenharia da Universidade do Porto}
\newcommand{\evm}{Eulerian Video Magnification}
\newcommand{\ica}{Independent Component Analysis}

% figure shorthand
%
\newcommand{\fig}[3][]{% position, filename, caption
\begin{figure}[#1]
  \begin{center}
    \leavevmode
    \includegraphics[width=\textwidth]{#2}
    \caption{#3}
    \label{fig:#2}
  \end{center}
\end{figure}}

% multiple line table cell
%
\newcommand{\multilinecell}[2][c]{%
  \begin{tabular}{@{}#1@{}}#2\end{tabular}}

%%----------------------------------------

%%========================================
%% Start of document
%%========================================
\begin{document}

%%----------------------------------------
%% Information about the work
%%----------------------------------------
\title{Android-based implementation of Eulerian Video Magnification for vital signs monitoring}
\author{Pedro Boloto Chambino}

%% Uncomment next line for date of submission
%\thesisdate{July 31, 2008}

%%Uncomment next line for copyright text if used
%\copyrightnotice{Name of the Author, 2008}

\supervisor{Supervisor}{Luís Teixeira}
\supervisor{Supervisor at Fraunhofer Portugal}{Luís Rosado}

%% Uncomment next line if necessary
%\supervisor{Second Supervisor}{Name of the Supervisor}

%% Uncomment committee stuff in the final version if used
%\committeetext{Approved in oral examination by the committee:}
%\committeemember{Chair}{Doctor Name of the President}
%\committeemember{External Examiner}{Doctor Name of the Examiner}
%\committeemember{Supervisor}{Doctor Name of the Supervisor}
%\signature

%% Specify cover logo (in folder ``figures'')
\logo{uporto-feup.pdf}

%% Uncomment next line for additional text  below the author's name (front page)
%\additionalfronttext{Preparação da Dissertação}

%%----------------------------------------
%% Preliminary materials
%%----------------------------------------

% remove unnecssary \include{} commands
\begin{Prolog}
  \chapter*{Abstract}

Here goes the abstract written in English.

Lorem ipsum dolor sit amet, consectetuer adipiscing elit. Sed vehicula
lorem commodo dui. Fusce mollis feugiat elit. Cum sociis natoque
penatibus et magnis dis parturient montes, nascetur ridiculus
mus. Donec eu quam. Aenean consectetuer odio quis nisi. Fusce molestie
metus sed neque. Praesent nulla. Donec quis urna. Pellentesque
hendrerit vulputate nunc. Donec id eros et leo ullamcorper
placerat. Curabitur aliquam tellus et diam. 

Ut tortor. Morbi eget elit. Maecenas nec risus. Sed ultricies. Sed
scelerisque libero faucibus sem. Nullam molestie leo quis
tellus. Donec ipsum. Nulla lobortis purus pharetra turpis. Nulla
laoreet, arcu nec hendrerit vulputate, tortor elit eleifend turpis, et
aliquam leo metus in dolor. Praesent sed nulla. Mauris ac augue. Cras
ac orci. Etiam sed urna eget nulla sodales venenatis. Donec faucibus
ante eget dui. Nam magna. Suspendisse sollicitudin est et mi. 

Fusce sed ipsum vel velit imperdiet dictum. Sed nisi purus, dapibus
ut, iaculis ac, placerat id, purus. Integer aliquet elementum
libero. Phasellus facilisis leo eget elit. Nullam nisi magna, ornare
at, aliquet et, porta id, odio. Sed volutpat tellus consectetuer
ligula. Phasellus turpis augue, malesuada et, placerat fringilla,
ornare nec, eros. Class aptent taciti sociosqu ad litora torquent per
conubia nostra, per inceptos himenaeos. Vivamus ornare quam nec sem
mattis vulputate. Nullam porta, diam nec porta mollis, orci leo
condimentum sapien, quis venenatis mi dolor a metus. Nullam
mollis. Aenean metus massa, pellentesque sit amet, sagittis eget,
tincidunt in, arcu. Vestibulum porta laoreet tortor. Nullam mollis
elit nec justo. In nulla ligula, pellentesque sit amet, consequat sed,
faucibus id, velit. Fusce purus. Quisque sagittis urna at quam. Ut eu
lacus. Maecenas tortor nibh, ultricies nec, vestibulum varius, egestas
id, sapien. 

Phasellus ullamcorper justo id risus. Nunc in leo. Mauris auctor
lectus vitae est lacinia egestas. Nulla faucibus erat sit amet lectus
varius semper. Praesent ultrices vehicula orci. Nam at metus. Aenean
eget lorem nec purus feugiat molestie. Phasellus fringilla nulla ac
risus. Aliquam elementum aliquam velit. Aenean nunc odio, lobortis id,
dictum et, rutrum ac, ipsum. 

Ut tortor. Morbi eget elit. Maecenas nec risus. Sed ultricies. Sed
scelerisque libero faucibus sem. Nullam molestie leo quis
tellus. Donec ipsum. Nulla lobortis purus pharetra turpis. Nulla
laoreet, arcu nec hendrerit vulputate, tortor elit eleifend turpis, et
aliquam leo metus in dolor. Praesent sed nulla. Mauris ac augue. Cras
ac orci. Etiam sed urna eget nulla sodales venenatis. Donec faucibus
ante eget dui. Nam magna. Suspendisse sollicitudin est et mi. 

Phasellus ullamcorper justo id risus. Nunc in leo. Mauris auctor
lectus vitae est lacinia egestas. Nulla faucibus erat sit amet lectus
varius semper. Praesent ultrices vehicula orci. Nam at metus. Aenean
eget lorem nec purus feugiat molestie. Phasellus fringilla nulla ac
risus. Aliquam elementum aliquam velit. Aenean nunc odio, lobortis id,
dictum et, rutrum ac, ipsum. 

Ut tortor. Morbi eget elit. Maecenas nec risus. Sed ultricies. Sed
scelerisque libero faucibus sem. Nullam molestie leo quis
tellus. Donec ipsum. 

\chapter*{Resumo}

O Resumo fornece ao leitor um sumário do conteúdo da dissertação.
Deverá ser breve mas conter detalhe suficiente e, uma vez que é a porta
de entrada para a dissertação, deverá dar ao leitor uma boa impressão
inicial.

Este texto inicial da dissertação é escrito no fim e resume numa
página, sem referências externas, o tema e o contexto do trabalho, a
motivação e os objectivos, as metodologias e técnicas empregues, os
principais resultados alcançados e as conclusões.

Este documento ilustra o formato a usar em dissertações na \Feup.
São dados exemplos de margens, cabeçalhos, títulos, paginação, estilos
de índices, etc. 
São ainda dados exemplos de formatação de citações, figuras e tabelas,
equações, referências cruzadas, lista de referências e índices.
%Este documento não pretende exemplificar conteúdos a usar. 
É usado texto descartável, \emph{Loren Ipsum}, para preencher a
dissertação por forma a ilustrar os formatos.

Seguem-se umas notas breves mas muito importantes sobre a versão 
provisória e a versão final do documento. 
A versão provisória, depois de verificada pelo orientador e de 
corrigida em contexto pelo autor, deve ser publicada na página 
pessoal de cada estudante/dissertação, juntamente com os dois 
resumos, em português e em inglês; deve manter a marca da água, 
assim como a numeração de linhas conforme aqui se demonstra.

A versão definitiva, a produzir somente após a defesa, em versão 
impressa (dois exemplares com capas próprias FEUP) e em versão 
eletrónica (6 CDs com "rodela" própria FEUP), deve ser limpa da marca de 
água e da numeração de linhas e deve conter a identificação, na primeira 
página, dos elementos do júri respetivo. 
Deve ainda, se for o caso, ser corrigida de acordo com as instruções 
recebidas dos elementos júri.

Lorem ipsum dolor sit amet, consectetuer adipiscing elit. Sed vehicula
lorem commodo dui. Fusce mollis feugiat elit. Cum sociis natoque
penatibus et magnis dis parturient montes, nascetur ridiculus
mus. Donec eu quam. Aenean consectetuer odio quis nisi. Fusce molestie
metus sed neque. Praesent nulla. Donec quis urna. Pellentesque
hendrerit vulputate nunc. Donec id eros et leo ullamcorper
placerat. Curabitur aliquam tellus et diam. 

Ut tortor. Morbi eget elit. Maecenas nec risus. Sed ultricies. Sed
scelerisque libero faucibus sem. Nullam molestie leo quis
tellus. Donec ipsum. Nulla lobortis purus pharetra turpis. Nulla
laoreet, arcu nec hendrerit vulputate, tortor elit eleifend turpis, et
aliquam leo metus in dolor. Praesent sed nulla. Mauris ac augue. Cras
ac orci. Etiam sed urna eget nulla sodales venenatis. Donec faucibus
ante eget dui. Nam magna. Suspendisse sollicitudin est et mi. 

Phasellus ullamcorper justo id risus. Nunc in leo. Mauris auctor
lectus vitae est lacinia egestas. Nulla faucibus erat sit amet lectus
varius semper. Praesent ultrices vehicula orci. Nam at metus. Aenean
eget lorem nec purus feugiat molestie. Phasellus fringilla nulla ac
risus. Aliquam elementum aliquam velit. Aenean nunc odio, lobortis id,
dictum et, rutrum ac, ipsum. 

Ut tortor. Morbi eget elit. Maecenas nec risus. Sed ultricies. Sed
scelerisque libero faucibus sem. Nullam molestie leo quis
tellus. Donec ipsum. 
 % the abstract
  \chapter*{Acknowledgements}

Aliquam id dui. Nulla facilisi. Nullam ligula nunc, viverra a, iaculis
at, faucibus quis, sapien. Cum sociis natoque penatibus et magnis dis
parturient montes, nascetur ridiculus mus. Curabitur magna ligula,
ornare luctus, aliquam non, aliquet at, tortor. Donec iaculis nulla
sed eros. Sed felis. Nam lobortis libero. Pellentesque
odio. Suspendisse potenti. Morbi imperdiet rhoncus magna. Morbi
vestibulum interdum turpis. Pellentesque varius. Morbi nulla urna,
euismod in, molestie ac, placerat in, orci. 

Ut convallis. Suspendisse luctus pharetra sem. Sed sit amet mi in diam
luctus suscipit. Nulla facilisi. Integer commodo, turpis et semper
auctor, nisl ligula vestibulum erat, sed tempor lacus nibh at
turpis. Quisque vestibulum pulvinar justo. Class aptent taciti
sociosqu ad litora torquent per conubia nostra, per inceptos
himenaeos. Nam sed tellus vel tortor hendrerit pulvinar. Phasellus
eleifend, augue at mattis tincidunt, lorem lorem sodales arcu, id
volutpat risus est id neque. Phasellus egestas ante. Nam porttitor
justo sit amet urna. Suspendisse ligula nunc, mollis ac, elementum
non, venenatis ut, mauris. Mauris augue risus, tempus scelerisque,
rutrum quis, hendrerit at, nunc. Nulla posuere porta orci. Nulla dui. 

Fusce gravida placerat sem. Aenean ipsum diam, pharetra vitae, ornare
et, semper sit amet, nibh. Nam id tellus. Etiam ultrices. Praesent
gravida. Aliquam nec sapien. Morbi sagittis vulputate dolor. Donec
sapien lorem, laoreet egestas, pellentesque euismod, porta at,
sapien. Integer vitae lacus id dui convallis blandit. Mauris non
sem. Integer in velit eget lorem scelerisque vehicula. Etiam tincidunt
turpis ac nunc. Pellentesque a justo. Mauris faucibus quam id
eros. Cras pharetra. Fusce rutrum vulputate lorem. Cras pretium magna
in nisl. Integer ornare dui non pede. 

\vspace{10mm}
\flushleft{The Name of the Author}
  % the acknowledgments
  \cleardoublepage
\thispagestyle{plain}

\vspace*{8cm}

\begin{flushright}
   \textsl{``You should be glad that bridge fell down. \\
           I was planning to build thirteen more to that same design''} \\
\vspace*{1.5cm}
           Isambard Kingdom Brunel
\end{flushright}
       % initial quotation if desired
  \cleardoublepage
  \pdfbookmark[0]{Table of Contents}{contents}
  %\setcounter{tocdepth}{1} % limit table of contents depth
  \tableofcontents
  \cleardoublepage
  \pdfbookmark[0]{List of Figures}{figures}
  \listoffigures
  \cleardoublepage
  \pdfbookmark[0]{List of Tables}{tables}
  \listoftables
  \chapter*{Abbreviations}
\chaptermark{ABBREVIATIONS}

\begin{flushleft}
\begin{tabular}{l p{0.8\linewidth}}
EVM      & \evm \\
ICA      & \ica \\
PPG      & Photo-plethysmography \\
FFT      & Fast Fourier transform \\
FPS      & Frames per second \\
JNI      & Java Native Interface \\
JVM      & Java Virtual Machine \\
OpenCV   & Open Source Computer Vision Library \\
IIR      & Infinite impulse response \\
SD       & Standard Deviation \\
\end{tabular}
\end{flushleft}
  % the list of abbreviations used
\end{Prolog}

%%----------------------------------------
%% Body
%%----------------------------------------
\StartBody

%% TIP: use a separate file for each chapter
\chapter{Introduction} \label{chap:intro}

\section*{}

% O primeiro capítulo da dissertação deve servir para apresentar o
% enquadramento e a motivação do trabalho e para identificar e
% definir os problemas que a dissertação aborda.

% Deve resumir as metodologias utilizadas no trabalho e termina
% apresentando um breve resumo de cada um dos capítulos
% posteriores.

This chapter introduces this work, by first presenting its context,
motivation, and project's objectives, on sections \ref{sec:intro:context},
\ref{sec:intro:motivation}, and \ref{sec:intro:objectives}, respectively.
Finally, section \ref{sec:intro:outline} describes the document outline.

\section{Context} \label{sec:intro:context}

% Esta secção descreve a área em que o trabalho se insere, podendo
% referir um eventual projeto de que faz parte e apresentar uma breve
% descrição da empresa onde o trabalho decorreu.

\evm{} is a method, recently presented at
\emph{SIGGRAPH}\footnote{\url{http://www.siggraph.org/}} 2012, capable of
revealing temporal variations in videos that are impossible to see
with the naked eye. Using this method, it is possible to visualize
the flow of blood as it fills the face~\cite{Wu2012Eulerian}.
Which provides enough information to assess the heart rate in a
contact-free way using a camera~\cite{Wu2012Eulerian,
Poh2010Non, Poh2011Advancements}.

The main field of this research work is \emph{image processing
and computer vision}, whose main purpose is to translate dimensional
data from the real world in the form of images into numerical
or symbolical information.

Other fields include \emph{medical applications}, \emph{software
development for mobile devices}, \emph{digital signal processing}.

This research work is an internal project of \emph{Fraunhofer
Portugal}\footnote{\url{http://www.fraunhofer.pt/}} supervised by
Luís Rosado. Fraunhofer Portugal a is non-profit private association
founded by Fraunhofer-Gesellschaft\footnote{\url{http://www.fraunhofer.de/en/about-fraunhofer/}}~\cite{Fraunhofer2013} and

\begin{quote}
  ``aims on the creation of scientific knowledge capable of
  generating added value to its clients and partners, exploring
  technology innovations oriented towards economic growth, the
  social well-being and the improvement of the quality of life of
  its end-users.''~\cite{Fraunhofer2013}
\end{quote}

\section{Motivation} \label{sec:intro:motivation}

% Apresenta a motivação e enumera os objetivos do trabalho terminando
% com um resumo das metodologias para a prossecução dos objetivos.

Due to being recently proposed, the \evm{} method implementation
has not been tested in smartphones yet.

There has been some successful effort on the assessment of vital
signs, such as, heart rate, and breathing rate, in a contact-free
way using a webcamera~\cite{Wu2012Eulerian, Poh2010Non, Poh2011Advancements},
and even a smartphone~\cite{Vitrox2013, Philips2013}.

Other similar products, which require specialist hardware and are
thus expensive, include \emph{laser Doppler}~\cite{Ulyanov1993Pulse},
\emph{microwave Doppler radar}~\cite{Greneker1997Radar}, and
\emph{thermal imaging}~\cite{Garbey2007Contact}.

Since it is a cheaper method of assessing vital signs in a
contact-free way than the above products, this research work has
potential for advancing fields, such as, \emph{telemedicine},
\emph{personal health-care}, and \emph{ambient assisting living}.

Despite the existence of very similar products by
\emph{Philips}~\cite{Philips2013} and
\emph{ViTrox Technologies}~\cite{Vitrox2013}
to the one proposed on this research work, the product to be developed
during this research work will have additional features, such as,
face tracking while assessing the heart rate, and augmented reality
to visualize the blood flow.

\section{Objectives} \label{sec:intro:objectives}

% Enumera os objetivos do trabalho terminando
% com um resumo das metodologias para a prossecução dos objetivos.

This research work goal is to test the feasibility of the
implementation of the \evm{} method on smartphones by developing
an \emph{Android} application for monitoring vital signs based on
the \evm{} method.

This application should include the following features:

\begin{itemize}
  \item heart rate detection and assessment based on the \evm{}
        method;
  \item display real-time changes, such as, the magnified blood
        flow, obtained from the \evm{} method;
  \item deal with artifacts' motion, due to, person and/or
        smartphone movement.
\end{itemize}

The application performance should then be evaluated through tests
with several individuals and the assessed heart rate compared to
the ones detected by another application~\cite{Vitrox2013, Philips2013},
and to the measurement of an electronic sphygmomanomete.

\pagebreak

\section{Outline} \label{sec:intro:outline}

% TODO rewrite outline

The rest of the document is structured as follows:

\begin{description}
  \item[Chapter~\ref{chap:sota}] introduces the concepts necessary to
        understand the presented problem. In addition, it presents
        the existing related work, and a description of the technologies
        to be used.
  \item[Chapter~\ref{chap:problem}] ...
  \item[Chapter~\ref{chap:solution}] presents the approach taken to
        solve the problem. Moreover, it introduces the testing and
        evaluation methodologies.
  \item[Chapter~\ref{chap:impl}] ...
  \item[Chapter~\ref{chap:results}] ...
  \item[Chapter~\ref{chap:conclusions}] ...
\end{description}

\chapter{State of the art} \label{chap:sota}

\section*{}

% Neste capítulo é descrito o estado da arte e são
% apresentados trabalhos relacionados para mostrar o que existe no
% mesmo domínio e quais os problemas em aberto.

% Deve deixar claro que existe uma oportunidade de desenvolvimento que
% cobre alguma falha concreta.

% O capítulo deve também efetuar uma revisão tecnológica às principais
% ferramentas utilizáveis no âmbito do projeto, justificando futuras
% escolhas.

% No final do capítulo deverá ser apresentado um resumo com as 
% principais conclusões que se podem tirar.

In this chapter, the background needed to understand the problem of 
heart rate detection using the \evm{} method is presented on 
section~\ref{sec:evm}. Before that, a similar research, which uses the 
\ica{} model to assess the cardiac pulse, is described in more 
detail on section~\ref{sec:relatedwork}. In addition, a description of the 
technologies to be used will be described on section~\ref{sec:technologies}.

\section{Related Work} \label{sec:relatedwork}

Successful attempts have already been made on real-time, contact-free 
heart rate assessment using a webcamera. In these attempts, two methods 
have been used: \ica{}, and \evm{}.

An \emph{iOS} application named \emph{Vital Signs Camera}, developed by 
\emph{Philips}, has also been able to detect a person's heart rate and 
breathing rate using the smartphone's camera~\cite{Philips2013}. 
However, as far as I know, there is no research work available to 
the public since its technology was developed internally by \emph{Philips}.

\subsection{\ica} \label{sec:ica}

\ica{} is a special case of \emph{blind source separation} and is a relatively 
new technique for uncovering independent signals from a set of observations 
that are composed of linear mixtures of the underlying 
sources~\cite{Comon1994Independent}.

In this case, the underlying source signal of interest is the cardiac pulse
that propagates throughout the body, which modify the path length of the 
incident ambient light due to volumetric changes in the facial blood vessels
during the cardiac cycle, such that subsequent changes in amount of reflected
light indicate the timing of cardiovascular events.

By recording a video of 
the facial region, the red, green, and blue (RGB) color sensors pick up a 
mixture of the reflected plethysmographic signal along with other sources of 
fluctuations in light due to artifacts. Each color sensor records a mixture 
of the original source signals with slightly different weights. These observed 
signals from the red, green and blue color sensors are denoted by $x_{1}(t)$,
$x_{2}(t)$ and $x_{3}(t)$ respectively, which are amplitudes of the recorded 
signals at time point $t$. In conventional \ica{} model the number of 
recoverable sources cannot exceed the number of observations, thus three 
underlying source signals were assumed, represented by $s_{1}(t)$, $s_{2}(t)$ 
and $s_{3}(t)$. The \ica{} model assumes that the observed signals are linear 
mixtures of the sources, i.e. $x_{i}(t) = \sum_{j=1}^{3} a_{ij} s_{j}(t)$ for 
each $i=1,2,3$. This can be represented compactly by the mixing equation

\begin{equation}
  x(t) = As(t)
\end{equation}

where the column vectors $x(t) = [x_{1}(t), x_{2}(t), x_{3}(t)]^{T}$, 
$s(t) = [s_{1}(t), s_{2}(t), s_{3}(t)]^{T}$ and the square $3\times3$ 
matrix $A$ contains the mixture coefficients $a_{ij}$. The aim of \ica{} model 
is to find a separating or demixing matrix $W$ that is an approximation of the 
inverse of the original mixing matrix $A$ whose output

\begin{equation}
  \hat{s}(t) = Wx(t)
\end{equation}

is an estimate of the vector $s(t)$ containing the underlying source signals. 
To uncover the independent sources, W must maximize the non-Gaussianity of 
each source. In practice, iterative methods are used to maximize or minimize 
a given cost function that measures non-Gaussianity~\cite{Poh2010Non, 
Poh2011Advancements}.

\section{\evm} \label{sec:evm}

In contrast to the \ica{} model that focus on extracting a single number,
the \evm{} uses localized spatial pooling and temporal filtering to extract
and reveal visually the signal corresponding to the cardiac pulse. This
allows for amplification and visualization of the heart rate signal at each 
location on the face. This creates potential for monitoring and diagnostic
applications to medicine, i.e. the asymmetry in facial blood flow can be a 
symptom of arterial problems.

Besides color amplification, the \evm{} method is also able to reveal 
low-amplitude motion which may be hard or impossible for humans to see.
Previous attempts to unveil imperceptible motions in videos have been 
made, such as, \cite{Liu2005Motion} which follows a \emph{Lagrangian}
perspective, as in fluid dynamics where the trajectory of particles
is tracked over time. By relying on accurate motion estimation and
additional techniques to produce good quality synthesis, such as, 
motion segmentation and image in-painting, the algorithm complexity 
and computation is expensive and difficult.

On the contrary, the \evm{} method is inspired by the \emph{Eulerian}
perspective, where properties of a voxel of fluid, such as pressure 
and velocity, evolve over time. The approach of this method to motion 
magnification is the exaggeration of motion by amplifying temporal 
color changes at fixed positions, instead of, explicitly estimation 
of motion.

This method approach, illustrated in figure~\ref{fig:evm}, combines 
spatial and temporal processing to emphasize subtle temporal changes 
in a video. First, the video sequence is decomposed into different 
spatial frequency bands. Because they may exhibit different 
signal-to-noise ratios, they may be magnified differently.
In the general case, the full Laplacian pyramid~\cite{Burt1983Laplacian}
may be computed. Then, temporal processing is performed on each 
spatial band. The temporal processing is uniform for all spatial 
bands, and for all pixels within each band. After that, the extracted
bandpass signal is magnified by a factor of $\alpha$, which can be 
specified by the user, and may be attenuated automatically. Finally, 
the magnified signal is added to the original and the spatial pyramid
collapsed to obtain the final output.

\fig{evm}{Overview of the \evm{} method.}

\subsection{Emphasize color variations for human pulse} \label{sec:evm-color}

The extraction of a person's cardiac pulse using the \evm{} method 
was demonstrated in~\cite{Wu2012Eulerian}. It was also presented that
using the right configuration can help extract the desired signal.
There are four steps to take then processing a video by \evm{}: 

\begin{enumerate}
  \item select a temporal bandpass filter;
  \item select an amplification factor, $\alpha$;
  \item select a spatial frequency cutoff (specified by spatial wavelength, 
        $\lambda_c$) beyond which an attenuated version of $\alpha$ is used;
  \item select the form of the attenuation for $\alpha$ —- either force 
        $\alpha$ to zero for all $\lambda < \lambda_c$, or linearly scale 
        $\alpha$ down to zero.
\end{enumerate}

For human pulse color variation, two temporal filters may be used, first
selecting frequencies within 0.4-4Hz, corresponding to 24-240 beats per 
minute (bpm), then a narrow band of 0.83-1Hz (50-60 bpm) may be used, 
if the extraction of the pulse rate was successful.

To emphasize the color change as much as possible, a large amplification 
factor, $\alpha \approx 100$, and spatial frequency cutoff, 
$\lambda_c \approx 1000$, is applied. With an attenuation of $\alpha$ to
zero for spatial wavelengths below $\lambda_c$.

The resulting output can be seen in figure~\ref{fig:evm-color}.

\fig{evm-color}{Emphasis of the face color changes using the \evm{} method.}

\section{Technologies} \label{sec:technologies}

Below are short descriptions of the main technologies that will be 
used during this research work.

\begin{description}

\item[Android SDK] \hfill 

\emph{Android SDK} is the development kit for the \emph{Android} 
platform. The \emph{Android} platform is an open source, Linux-based 
operating system, primarily designed for touchscreen mobile devices, 
such as, smartphones.

Because of its open source code and permissive licensing, it allows 
the software to be freely modified and distributed. This have allowed
\emph{Android} to be the software of choice for technology companies
who require a low-cost, customizable, and lightweight operating system 
for mobile devices and others.

\emph{Android} has also become the world's most widely used smartphone 
platform with a worldwide smartphone market share of 75\%
during the third quarter of 2012~\cite{Idc2013Android}.

\pagebreak

\item[OpenCV -- Computer Vision Library] \hfill 

\emph{OpenCV} is a library of programming functions mainly aimed at 
real-time image processing. To support these, it also includes a 
statistical machine learning library. Moreover, it is a cross-platform 
and open source library that is free to use and modify under the BSD 
license.

\begin{quote}
  ``OpenCV was built to provide a common infrastructure for computer 
  vision applications and to accelerate the use of machine perception 
  in the commercial products.''~\cite{Opencv2013About}
\end{quote}

\end{description}

\chapter{Approach} \label{chap:approach}

\section*{}

In this chapter, a simple description of the approach to be taken during 
this research work is presented on section~\ref{sec:approach}. Moreover, 
the solution perspective and the difficulties and problems that may arise 
are described on section~\ref{sec:solution} and~\ref{sec:difficulties},
respectively. The last section~\ref{sec:evaluation}, describes the 
evaluation process.

\section{Overview} \label{sec:approach}

% Incremental implementation of the application...

The approach or methodology taken to reach this research work objectives
is an incremental implementation of the application features or requirements.
It should be taken into consideration that this approach plan may be modified 
during the duration of the research work and is not a strict line of action.

Starting with the creation of a simple \emph{Android} application that uses
the \emph{OpenCV} library. Follows the implementation of face detection using 
the \emph{OpenCV} library, and the \evm{} method showing the resulting video 
in real-time. Then, after the implementation of a simple heart rate detection 
algorithm based on the \evm{} method, the evaluation of application will
be executed. This evaluation should then be executed every time a significant 
modification to the heart rate detection algorithm is made.

\section{Solution Perspective} \label{sec:solution}

% Explain the solution and why it was chosen...

To extract the cardiac pulse from a person's face, first, a face must be  
discovered on the input video, for that a simple \emph{OpenCV} face detection
algorithm will be used.

Then, in order to obtain the heart rate signal, 
there is the need to focus on an area to extract that signal. However,
the amplitude variation of the signal of interest is often must smaller 
than the noise inherent in the video. To enhance these subtle signals 
spatial polling can be used. Despite of that if the spatial polling 
applied is not large enough, the signal of interest will not be revealed.
Retrieved from the article~\cite{Wu2012Eulerian}, the equation~\ref{eq:noise}
gives an estimate for the size of the spatial polling need to revel the 
signal of interest at a certain noise power level.

\begin{equation} \label{eq:noise}
  S(r) = \sigma'^2 = k \frac{\sigma^2}{r^2}
\end{equation}

Where $\sigma^2$ is the noise power level, which can be estimated by using 
a technique as in~\cite{Liu2006Noise}, and $S(r)$ is the signal power of
such spatial polling filter. Thus, since the filtered noise power level,
$\sigma'^2$, is inversely proportional to $r^2$, it is possible to solve the 
equation~\ref{eq:noise} for $r$, where $k$ is a constant that depends on the shape of the spatial filter.

This area of interest can then be tracked using a simple \emph{OpenCV} 
feature tracking algorithm to deal with artifacts' motion.

In addition, the \evm{} method can be configured to amplify the color
variation as explained on section~\ref{sec:evm-color}.

The signal extracted should be recognizable as a cardiac pulse signal,
however further processing may be needed to detrend the signal. Also, 
since the pulse computation may be affected by noise, historical estimations
to reject artifacts may be implemented as in~\cite{Poh2010Non}.

To create the \emph{augmented reality effect} of the blood flow, 
the raw \evm{} method result is added to the input video.

An important part of the development is focused on the complete 
implementation of the \evm{} since the available framework is not 
destined to be used in real-time nor on smartphones.

\section{Difficulties} \label{sec:difficulties}

% Difficulties and how to overcome those...

During the course of this research work a number of possible problems may
occur which will hinder the development of the project and research. Some
of these difficulties have been predicted and a few possible solutions to 
those will be described below.

\begin{itemize}

\item
\textbf{Problem}
Noise created by smartphone and person motion, and by lighting changes.

Every small movement may heavily influence the method result since it
also amplifies motion. Thus, too much noise may obfuscate the cardiac 
pulse intended to be detected.

\textbf{Solution Perspective}
Noise reduction may be accomplished by using face detection (and feature 
tracking) to focus on a small area on the person's face that is independent
enough to not suffer large intensity variations due to the identified 
artifacts. In addition, historical estimations to reject artifacts may be
implemented and specific configuration to the \evm{} method to 
emphasize the color change as much as possible~\cite{Wu2012Eulerian} as
explained on section~\ref{sec:evm-color}.

\pagebreak

\item
\textbf{Problem}
Smartphone computing power may not be enough for real-time processing.

The \evm{} method is able to run in real-time~\cite{Wu2012Eulerian}. However, 
the lower computing power of smartphones and extra processing, such as, 
face detection and tracking, noise reduction techniques, and signal 
normalization algorithms, may cause the method to not be able to execute 
in real-time.

\textbf{Solution Perspective}
If this problem arises, a possible solution is to reduce the computing power
required, by switching to different approaches or even the removal of 
some features, such as, feature tracking.

\end{itemize}

\section{Evaluation} \label{sec:evaluation}

% The evaluation should compare results and benchmark the algorithms...

The evaluation main goal is the validation of the cardiac pulse assessed.
This may be accomplished, by comparing the signal and average of the heart 
rate detected by the application with the one captured by an heart rate monitor
or a pulse oximeter. These measurements should be done simultaneously so they 
can be correctly compared.

In addition, these measurements will be obtained in different settings, 
such as, indoor, outdoor, with face motion, with face tilting, while the 
person is speaking, during lighting variations, to verify the application 
robustness.

At the end, the application will also be compared against the one developed
by \emph{Philips} -- Vital Signs Camera. The comparison will focus on both:
average heart rate detected, and application performance.

\chapter{Work Plan} \label{chap:workplan}

\section*{}

This chapter describes the tasks to be accomplished during the
following 5 months. Each task will be subsequently divided into
subtasks and provide an estimated time.

The calendatization of the tasks during the next 5 months will
be presented in the section~\ref{sec:schedule}.

\section{Tasks} \label{sec:tasks}

\begin{description}

\item[Literature Review] \hfill

In order to implement key features of this research work, several
concepts and algorithms need to be deeper analysed and studied.

\begin{itemize}
  \item Explore \emph{OpenCV} face detection algorithms.
  \item Explore \evm{} method for heart rate detection.
  \item Explore \emph{OpenCV} feature detection and tracking algorithms.
  \item Explore noise reduction to improve heart rate detection
        using the \evm{} method.
\end{itemize}

Estimated time: 4 weeks.

\pagebreak

\item[Implementation] \hfill

Implement an \emph{Android} application for detecting a person
heart rate using the smartphone's camera.

\begin{itemize}
  \item Integrate \emph{Android} and \emph{OpenCV} library.
  \item Implement face detection using \emph{OpenCV} library.
  \item Implement real-time \emph{Augmented Reality} using
        \evm{} method.
  \item Implement heart rate detection based on the values
        obtained by the \evm{} method.
  \item Implement feature detection and tracking using \emph{OpenCV} library.
  \item Implement noise reduction solutions explored.
  \item Improve application design and user experience.
\end{itemize}

Estimated time: 10 weeks.

\item[Evaluation] \hfill

Compare the heart rate detected to the one obtained from the
\emph{Philips}~\cite{Philips2013} application -- \emph{Vital Signs
Camera} -- and to the measurement of an heart rate monitor or pulse
oximeter in different settings.

Estimated time: 4 weeks.

\item[Thesis Writing] \hfill

Thesis writing, including a detailed description of the algorithms
and methods used, and the results obtained.

Estimated time: 6 weeks.

\end{description}

\pagebreak

\section{Schedule} \label{sec:schedule}

Figure~\ref{fig:gantt} is a Gantt diagram containing the schedule
of the previously defined tasks for the next 5 months.

Tasks \emph{Literature Review} and \emph{Implementation} overlap
because important parts of the implementation will be heavily
influenced by a few articles. And, \emph{Implementation} and
\emph{Evaluation} overlap because the evaluation phase may provide
further insight for improving the implemented application and
algorithms.


%%----------------------------------------
%% Final materials
%%----------------------------------------

%% Bibliography
%% Comment the next command if BibTeX file not used
%% bibliography is in ``myrefs.bib''
\PrintBib{myrefs}

%% comment next 2 commands if numbered appendices are not used
\appendix
\chapter{Loren Ipsum} \label{ap1:loren}

Depois das conclusões e antes das referências bibliográficas,
apresenta-se neste anexo numerado o texto usado para preencher a
dissertação.

\section{O que é o \emph{Loren Ipsum}?}

\emph{\textbf{Lorem Ipsum}} is simply dummy text of the printing and
typesetting industry. Lorem Ipsum has been the industry's standard
dummy text ever since the 1500s, when an unknown printer took a galley
of type and scrambled it to make a type specimen book. It has survived
not only five centuries, but also the leap into electronic
typesetting, remaining essentially unchanged. It was popularised in
the 1960s with the release of Letraset sheets containing Lorem Ipsum
passages, and more recently with desktop publishing software like
Aldus PageMaker including versions of Lorem Ipsum~\citep{kn:Lip08}. 

\section{De onde Vem o Loren?}

Contrary to popular belief, Lorem Ipsum is not simply random text. It
has roots in a piece of classical Latin literature from 45 BC, making
it over 2000 years old. Richard McClintock, a Latin professor at
Hampden-Sydney College in Virginia, looked up one of the more obscure
Latin words, consectetur, from a Lorem Ipsum passage, and going
through the cites of the word in classical literature, discovered the
undoubtable source. Lorem Ipsum comes from sections 1.10.32 and
1.10.33 of ``de Finibus Bonorum et Malorum'' (The Extremes of Good and
Evil) by Cicero, written in 45 BC. This book is a treatise on the
theory of ethics, very popular during the Renaissance. The first line
of Lorem Ipsum, ``Lorem ipsum dolor sit amet\ldots'', comes from a line in
section 1.10.32.

The standard chunk of Lorem Ipsum used since the 1500s is reproduced
below for those interested. Sections 1.10.32 and 1.10.33 from ``de
Finibus Bonorum et Malorum'' by Cicero are also reproduced in their
exact original form, accompanied by English versions from the 1914
translation by H. Rackham.

\section{Porque se usa o Loren?}

It is a long established fact that a reader will be distracted by the
readable content of a page when looking at its layout. The point of
using Lorem Ipsum is that it has a more-or-less normal distribution of
letters, as opposed to using ``Content here, content here'', making it
look like readable English. Many desktop publishing packages and web
page editors now use Lorem Ipsum as their default model text, and a
search for ``lorem ipsum'' will uncover many web sites still in their
infancy. Various versions have evolved over the years, sometimes by
accident, sometimes on purpose (injected humour and the like). 

\section{Onde se Podem Encontrar Exemplos?}

There are many variations of passages of Lorem Ipsum available, but
the majority have suffered alteration in some form, by injected
humour, or randomised words which don't look even slightly
believable. If you are going to use a passage of Lorem Ipsum, you need
to be sure there isn't anything embarrassing hidden in the middle of
text. All the Lorem Ipsum generators on the Internet tend to repeat
predefined chunks as necessary, making this the first true generator
on the Internet. It uses a dictionary of over 200 Latin words,
combined with a handful of model sentence structures, to generate
Lorem Ipsum which looks reasonable. The generated Lorem Ipsum is
therefore always free from repetition, injected humour, or
non-characteristic words etc. 


%% Index
%% Uncomment next command if index is required
%% don't forget to run ``makeindex mieic-en'' command
%\PrintIndex

\end{document}
