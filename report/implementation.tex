\chapter{Implementation details} \label{chap:impl}

\section*{}

% Este capítulo pode ser dedicado à apresentação de detalhes de nível
% mais baixo relacionados com o enquadramento e implementação das
% soluções preconizadas no capítulo anterior.
% Note-se no entanto que detalhes desnecessários à compreensão do
% trabalho devem ser remetidos para anexos.

This chapter ...
% TODO

\section{Overview} \label{sec:sol:overview}

\fig{algorithm-flow}{Implemented library modules overview.}

In order to create an Android application capable of estimating a person's
heart rate, a desktop application was developed due to its increase in
implementation speed and easier testing. Later, the main part of this
application was integrated into an Android application.

The application was divided into several modules, illustrated in
figure~\ref{fig:algorithm-flow}, which could be extracted into an independent
library. The language used to implement the desktop application and library
were C/C++. In addition, for the image processing operations, the
computer vision library, OpenCV, was used.

% TODO explain the purpose of some of the modules

\section{\evm{} implementations} \label{sec:impl:evm}

\fig{evm-flow}{\evm{} method steps.}

This section presents the details of several different implementations of the
\evm{} method.

The first implementations, described on sections~\ref{sec:impl:evm:gdownideal},
\ref{sec:impl:evm:gdowniir} and~\ref{sec:impl:evm:lpyriir},
were developed in Java to facilitate the integration into the Android
application. However, the OpenCV Java binding was still in its early stages
which end up creating difficulties for the development. Thus, the final
implementation, on section~\ref{sec:impl:evm:final}, was implemented in C/C++,
which also reduces the number of JNI calls from the Android JVM and
increases the application performance.

The purpose of implementing multiple variants of the method was to study
how the method worked and select which spatial and temporal filter would
lead to better results.

\subsection{EvmGdownIdeal} \label{sec:impl:evm:gdownideal}

\subsection{EvmGdownIIR} \label{sec:impl:evm:gdowniir}

\subsection{EvmLpyrIIR} \label{sec:impl:evm:lpyriir}

\subsection{Performance optimized EvmGdownIIR} \label{sec:impl:evm:final}

\section{Face detection stabilization} \label{sec:impl:face}

% TODO face detection stabilization

\section{Signal validations} \label{sec:impl:validations}

% TODO raw signal noise
% TODO Pulse wave detection algorithm simplification

\section{Heart rate estimation} \label{sec:impl:estimation}

% TODO power spectrum gave better results because peak count was not very consistent

\section{Performance optimizations} \label{sec:impl:performance}

% TODO EVM optimizations
% TODO EVM on face box only
% TODO face detection every 1 second
% TODO only deal with RGB channels instead of RGBA since alpha was constant

\section{Android integration} \label{sec:impl:android}

\fig{android-flow}{Integration workflow between Android native and Java parts.}

% TODO usage of JNI and Android NDK
% TODO reimplementation of OpenCV Android:
%      RGB frame, zoom stretch, rotation, camera switch, fps switch

\section{Chapter summary}

% TODO
