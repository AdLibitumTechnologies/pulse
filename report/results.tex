\chapter{Results} \label{chap:results}

\section*{}

% Dependendo do volume, a avaliação do trabalho pode ser incluída no
% capítulo anterior ou pode constituir um capítulo separado.

This chapter... % TODO

\section{Performance} \label{sec:results:perf}

In order to improve the algorithm and application performance, metrics were
obtained using the \emph{High performance C++ Profiler}~\cite{Andrew2013High}.
This profiler was chosen because of its performance claim and easy integration
in the project.

The machine specifications on which all metrics were obtained are:

\begin{description}
  \item[Machine] MacBook Pro, 15-inch, Late 2008
  \item[Processor] 2.53 GHz Inter Core 2 Duo
  \item[Memory] 4 GB 1067 MHz DDR3
  \item[Operating System] OS X Mountain Lion
\end{description}

Below are the dates of the various performance milestones accomplished.
For each date there are two tables with the respective metrics that can be found
on appendix~\ref{appx:perf}. The metrics are represented in \emph{CPU cycles}
obtained by using the instruction \texttt{rdtsc}~\cite{Andrew2013High}.

\begin{description}
  \item \textbf{Mar 20}, tables on page~\pageref{pdf:profile:1}\hfill\\
        These metrics match the initial implementation of the C/C++ version
        of the desktop application when performance profiling was integrated.
        The important functions found on the first table are:

        \begin{itemize}
          \item \texttt{Pulse::onFrame}, which is the function that detects a
                person face and estimates the heart rate for each face detected.
                Currently taking approximately 52 million cycles;
          \item \texttt{EvmGdownIIR::onFrame}, which is the function that
                implements the \evm{} method. Currently taking approximately
                28 million cycles.
        \end{itemize}

        The operations taking more CPU cycles were easily identified by the
        second table on~\ref{pdf:profile:1}, also represented on
        table~\ref{tab:profile:1}:

        \begin{table}[h]
          \centering
          \begin{tabular}{|l|r|r|r|}
            \hline
            Operation & Calls & Self MCycles & Self Avg \\
            \hline
            wait key                                & 301 & 9892.1852 (32\%) & 32.8644 \\
            pyrUp                                   & 300 & 3472.5267 (11\%) & 11.5751 \\
            detect faces                            & 301 & 2941.3343 (10\%) & 9.7719  \\
            imshow                                  & 301 & 2555.0751 ( 8\%) & 8.4886  \\
            capture                                 & 302 & 2268.6110 ( 7\%) & 7.5120  \\
            pyrDown                                 & 301 & 1988.4146 ( 7\%) & 6.6060  \\
            convert to 8 bit                        & 300 & 1603.1581 ( 5\%) & 5.3439  \\
            void cv::detrend(...)                   & 301 & 1538.8180 ( 5\%) & 5.1124  \\
            resize face box                         & 301 & 1270.4278 ( 4\%) & 4.2207  \\
            resize and draw face box back to frame  & 301 & 1214.4998 ( 4\%) & 4.0349  \\
            add back to original frame              & 300 & 811.3654  ( 3\%) & 2.7046  \\
            convert to float                        & 301 & 604.0527  ( 2\%) & 2.0068  \\
            \hline
          \end{tabular}
          \caption{
            Operations taking more CPU cycles on the initial implementation of
            the C/C++ version of the desktop application when performance
            profiling was added.
          }
          \label{tab:profile:1}
        \end{table}

  \item \textbf{Mar 26} tables on page~\pageref{pdf:profile:2}\hfill\\
        ...

  \item \textbf{Apr 3} tables on page~\pageref{pdf:profile:3}\hfill\\
        ...

  \item \textbf{May 23} tables on page~\pageref{pdf:profile:4}\hfill\\
        ...
\end{description}

% TODO EVM optimizations
% TODO EVM on face box only
% TODO face detection every 1 second
% TODO only deal with RGB channels instead of RGBA since alpha was constant

\section{Heart rate comparison} \label{sec:results:heart}

% TODO procedure

\section{Chapter summary}

% TODO
