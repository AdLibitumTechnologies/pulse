\chapter{Results} \label{chap:results}

\section*{}

% Dependendo do volume, a avaliação do trabalho pode ser incluída no
% capítulo anterior ou pode constituir um capítulo separado.

This chapter... % TODO

\section{Performance} \label{sec:results:perf}

Smartphone devices have a low computational power and this was one of the
main difficulties. The first implementations of the application were not
fast enough and the implemented Android application running on a
\emph{HUAWEI MediaPad} tablet with the Android version 4.0.3 was only capturing
around 2 frames per second.

In order to improve the algorithm and application performance, metrics were
obtained using the \emph{High performance C++ Profiler}~\cite{Andrew2013High}.
This profiler was chosen because of its performance claim and easy integration
in the project.

The machine specifications on which all metrics were obtained are:

\begin{description}
  \item[Machine] MacBook Pro, 15-inch, Late 2008
  \item[Processor] 2.53 GHz Inter Core 2 Duo
  \item[Memory] 4 GB 1067 MHz DDR3
  \item[Operating System] OS X Mountain Lion
\end{description}

The input video used was the video \emph{face} found at~\cite{Wu2013Eulerian}.

On appendix~\ref{appx:perf} various tables can be found representing
several performance boosts accomplished. The unit used by the
\emph{High performance C++ Profiler} is the number of CPU cycles spent by
that operation, obtained by using the instruction
\texttt{rdtsc}~\cite{Andrew2013High}.

The main functions that needed to be optimized and analyzed are:

\begin{itemize}
  \item \texttt{Pulse::onFrame}, which is the function that detects a
        person face and estimates the heart rate for each face detected;
  \item \texttt{EvmGdownIIR::onFrame}, which is the function that
        implements the \evm{} method.
\end{itemize}

Table~\ref{tab:profile:initial} shows that these functions were spending
a big percentage of the CPU cycles, with the \texttt{Pulse} function
taking 51\% and the \evm{} method occupying 28\%.

\begin{table}[p]
  \centering
  \begin{tabular}{lrrr}
    \hline
    Operation & Calls & MCycles & Avg \\
    \hline
    Pulse::onFrame        & 301 & 15630.5432 ( 51\%) & 51.9287 \\
    EvmGdownIIR::onFrame  & 301 &  8507.6548 ( 28\%) & 28.2646 \\
    \hline
  \end{tabular}
  \caption{
    Performance metrics obtained form the initial implementation of the C/C++
    version of the desktop application. Excerpt from~\ref{pdf:profile:1}.
  }
  \label{tab:profile:initial}
\end{table}

The operations taking more CPU cycles were easily identified by the
second table on~\ref{pdf:profile:1}, also represented on
table~\ref{tab:profile:top}. These operations are:

\begin{itemize}
  \item OpenCV HighGUI module functions: \emph{wait key}, \emph{imshow},
        \emph{capture};
  \item resize operations: \emph{pyrUp}, \emph{pyrDown}, \emph{resize face box},
        and \emph{resize and draw face box back to frame};
  \item conversion operations: \emph{convert to 8 bit}, \emph{convert to float};
  \item matrix operations: \emph{add back to original frame};
  \item \emph{detect faces}, is the OpenCV object detector, details on
        section~\ref{sec:impl:face};
  \item \emph{detrend}, is the detrending method explained on
        section~\ref{sec:sota:post:detrend};
\end{itemize}

\begin{table}
  \centering
  \begin{tabular}{lrrr}
    \hline
    Operation & Calls & Self MCycles & Self Avg \\
    \hline
    wait key                                & 301 & 9892.1852 (32\%) & 32.8644 \\
    pyrUp                                   & 300 & 3472.5267 (11\%) & 11.5751 \\
    detect faces                            & 301 & 2941.3343 (10\%) & 9.7719  \\
    imshow                                  & 301 & 2555.0751 ( 8\%) & 8.4886  \\
    capture                                 & 302 & 2268.6110 ( 7\%) & 7.5120  \\
    pyrDown                                 & 301 & 1988.4146 ( 7\%) & 6.6060  \\
    convert to 8 bit                        & 300 & 1603.1581 ( 5\%) & 5.3439  \\
    detrend                                 & 301 & 1538.8180 ( 5\%) & 5.1124  \\
    resize face box                         & 301 & 1270.4278 ( 4\%) & 4.2207  \\
    resize and draw face box back to frame  & 301 & 1214.4998 ( 4\%) & 4.0349  \\
    add back to original frame              & 300 & 811.3654  ( 3\%) & 2.7046  \\
    convert to float                        & 301 & 604.0527  ( 2\%) & 2.0068  \\
    \hline
  \end{tabular}
  \caption{
    Operations taking more CPU cycles on the initial implementation of
    the C/C++ version of the desktop application when performance
    profiling was added. Excerpt from~\ref{pdf:profile:1}.
  }
  \label{tab:profile:top}
\end{table}

The first performance boost was achieved by replacing the resize operations
by faster resize methods. All together the resize operations were spending
26\% of the CPU cycles as shown by table~\ref{tab:profile:top}.
The optimizations for the operations \emph{pyrUp} and
\emph{pyrDown} are detailed on section~\ref{sec:impl:evm:final}.
The rest of the resize operations were changed to use the OpenCV
nearest neighbor interpolation method since the difference of size was small.
These optimizations reduced the total percentage of the CPU cycles
to 7\% as shown on table~\ref{tab:profile:resize}.

\begin{table}
  \centering
  \begin{tabular}{lrrr}
    \hline
    Operation & Calls & Self MCycles & Self Avg \\
    \hline
    pyrUp                                   & 300 & 583.2825 (2\%) & 1.9443 \\
    pyrDown                                 & 301 & 510.0373 (2\%) & 1.6945 \\
    resize and draw face box back to frame  & 301 & 430.7046 (2\%) & 1.4309 \\
    resize face box                         & 301 & 307.4483 (1\%) & 1.0214 \\
    \hline
  \end{tabular}
  \caption{
    Performance improvement on resize operations by using faster interpolations
    methods. Excerpt from~\ref{pdf:profile:2}.
  }
  \label{tab:profile:resize}
\end{table}

The second performance boost was accomplished by reducing the number of
executions of the OpenCV object detector. This operation was occupying
around 10\% of the CPU cycles as shown by table~\ref{tab:profile:top}.
By executing the face detection operation only every 10 frames its number
of CPU cycles was greatly reduced to 1\% as shown on
table~\ref{tab:profile:face}.

\begin{table}
  \centering
  \begin{tabular}{lrrr}
    \hline
    Operation & Calls & Self MCycles & Self Avg \\
    \hline
    detect faces &  30 & 288.2041 (1\%) & 9.6068 \\
    \hline
  \end{tabular}
  \caption{
    Performance improvement on face detection by reducing the number of times
    the OpenCV object detector was executed. Excerpt from~\ref{pdf:profile:3}.
  }
  \label{tab:profile:face}
\end{table}

A last performance boost applied to the implemented algorithm was the removal of
the resize operations \emph{resize face box} and
\emph{resize and draw face box back to frame} by modifying the \evm{} method
implemented and removing the requirement of the input image to be always of the
same size.

The result of these performance improvements are shown on
table~\ref{tab:profile:final}. With the \texttt{Pulse} function spending
only 14\% of the CPU cycles and a performance improvement of 37\%,
and the \evm{} method occupying 6\% of the CPU cycles and a performance
improvement of 22\%.

\begin{table}
  \centering
  \begin{tabular}{lrrr}
    \hline
    Operation & Calls & MCycles & Avg \\
    \hline
    Pulse::onFrame       & 300 & 4215.0788 ( 14\%) & 14.0503 \\
    EvmGdownIIR::onFrame & 287 & 1818.3953 (  6\%) &  6.3359 \\
    \hline
  \end{tabular}
  \caption{
    Final performance metrics of the main functions.
    Excerpt from~\ref{pdf:profile:4}.
  }
  \label{tab:profile:final}
\end{table}

This allowed for the implemented Android application to run at approximately
15 frames per second on the same Android device previously mentioned.

\pagebreak

\section{Heart rate comparison} \label{sec:results:heart}

% TODO procedure

blah blah

\begin{table}
  \centering
  \begin{tabular}{c*{9}{r}}
    \hline
      \multirow{2}{*}{Subject} &
      \multicolumn{3}{c}{Normal} &
      \multicolumn{3}{c}{Normal} &
      \multicolumn{3}{c}{After exercise} \\
    \cline{2-10}
      & Pulse & Sphy & Error
      & ViTrox & Sphy & Error
      & Pulse & Sphy & Error \\
    \hline
      A & 55 & 57 & 0.04 & 66 & 61 & 0.08 & 54 &  87 & 0.38 \\
      B & 60 & 60 & 0.00 & 61 & 57 & 0.07 & 66 &  65 & 0.02 \\
      C & 57 & 76 & 0.25 & 72 & 72 & 0.00 & 57 &  98 & 0.42 \\
      D & 54 & 66 & 0.18 & 65 & 63 & 0.03 & 48 &  71 & 0.32 \\
      E & 47 & 75 & 0.37 & 81 & 77 & 0.05 & 64 &  84 & 0.24 \\
      F & 54 & 57 & 0.05 & 55 & 56 & 0.02 & 51 &  63 & 0.19 \\
      G & 55 & 84 & 0.35 & 87 & 87 & 0.00 & 69 & 102 & 0.32 \\
      H & 62 & 83 & 0.25 & 79 & 82 & 0.04 & 66 & 105 & 0.37 \\
      I & 56 & 59 & 0.05 & 55 & 54 & 0.02 & 47 &  80 & 0.41 \\
      J & 55 & 58 & 0.05 &    &    &      &    &     &      \\
    \hline
      Average & 55.50 & 67.50 & 0.16 & 69.00 & 67.67 & 0.03 & 58.00 & 83.89 & 0.30 \\
      Median  & 55.00 & 63.00 & 0.12 & 66.00 & 63.00 & 0.03 & 57.00 & 84.00 & 0.32 \\
      StdDev  &  3.98 & 10.97 & 0.14 & 11.50 & 12.19 & 0.03 &  8.46 & 15.64 & 0.13 \\
    \hline
  \end{tabular}
  \caption{}
  \label{}
\end{table}

\begin{table}
  \centering
  \begin{tabular}{crrr}
    \hline
      \multirow{2}{*}{Subject} &
      \multicolumn{3}{c}{Normal} \\
    \cline{2-4}
      & Pulse & Sphy & Error \\
    \hline
      A & 55 & 57 & 0.04 \\
      B & 60 & 60 & 0.00 \\
      D & 54 & 66 & 0.18 \\
      F & 54 & 57 & 0.05 \\
      I & 56 & 59 & 0.05 \\
      J & 55 & 58 & 0.05 \\
    \hline
      Average & 55.67 & 59.50 & 0.06 \\
      Median  & 55.00 & 58.50 & 0.05 \\
      StdDev  &  2.25 &  3.39 & 0.06 \\
    \hline
  \end{tabular}
  \caption{only bpm lower than 70}
  \label{}
\end{table}

\section{Chapter summary}

% TODO
