\chapter{Conclusions} \label{chap:conclusions}

\section*{}

% Deve ser apresentado um resumo do trabalho realizado e apreciada a
% satisfação dos objetivos do trabalho, uma lista de contribuições
% principais do trabalho e as direções para trabalho futuro.

% A escrita deste capítulo deve ser orientada para a total compreensão
% do trabalho, tendo em atenção que, depois de ler o Resumo e a
% Introdução, a maioria dos leitores passará à leitura deste capítulo de
% conclusões e recomendações para trabalho futuro.

This final chapter presents a review of the relevant information obtained
from this work and an exposition of further work and research.

Section~\ref{sec:conclusions:objectives} gives an overall description of
the work done, from the performance improvements of the \evm{} method,
to the creation of the Android application capable of estimating a
person's heart rate using the device's camera.

Finally, section~\ref{sec:conclusions:future} exposes future work that
could follow the development of an Android-based implementation
of the \evm{}.

\section{Objective satisfaction} \label{sec:conclusions:objectives}

The main goal of this work was providing an \evm{}-based method capable of
running on an Android device. To achieve that, various real-time
implementations of the \evm{} method were developed with the aid of the
image processing library OpenCV. However, these were
not efficient enough to execute on a smartphone in real-time. Hence,
a performance profiler was integrated into a desktop application, in order to
increase the performance of the application and the \evm{} method,
which was occupying 28\% of the total CPU cycles used by the application.

This \evm{} method implementation was using a temporal bandpass filter composed
by subtracting two first-order IIR lowpass filters, which is more convenient
for a real-time implementation than an ideal temporal filter implemented by
applying the Fourier transform to each pixel for a video segment.

The main performance boost was accomplished by replacing the multiple
operations of Gaussian blurring and downsampling, suggested
by~\cite{Wu2012Eulerian}, by a single resize operation using the OpenCV
interpolation method \emph{AREA}, which produces a similar result of the
Gaussian filter. Moreover, the inverse operations of multiple
upsampling and Gaussian blurring followed by an image resize
for cases when the downsampled image size had to be rounded was also
replaced by a single resize operation using the linear interpolation method.

At the end, the \evm{} method implemented was occupying only 6\% of the total
CPU cycles used by the application, which corresponds to a performance
improvement of 22\% in comparison to the initial implementation.

As an example of application of the developed \evm{} method, an Android
application, named Pulse, was implemented which was capable of estimating a
person's heart rate by capturing and analyzing that person's PPG signal.

Since the implemented algorithm was developed in the programming language
C/C++ for performance reasons, the integration into the Android platform
was done through the use of JNI and the Android NDK.

The application workflow started by grabbing an image from the device's
camera. A person's face was detected using the OpenCV object detect module
which was previously trained to detect human faces. A region of interest (ROI)
of the person's face would then be fed into the implemented \evm{} method to
amplify color variations. The average of the ROI green channel was computed,
in order to increase the signal-to-noise ratio, and stored. Along the time,
these stored values represent a PPG signal of the underlying blood flow
variations. The signal is further processed using the detrend method
to remove trends from the signal without magnitude distortion. It is then
validated as a cardiac pulse signal by detecting its peaks in order to
verify its shape and timing. Finally, the heart rate estimation is
computed by identifying the frequency with the higher power spectrum of
the signal.

In order to validate the heart rate estimations obtained by the implemented
Android application, Pulse, measurements from 9 participants were compared
to the readings of a sphygmomanometer. The agreement between the two
according to the Bland-Altman plots analysis had a mean bias of $-12.00$ with
95\% limits of agreement $-33.13$ to $9.13$ bpm. Heart rate
measurements above 70 bpm according to the sphygmomanometer were not being
correctly estimated by the Pulse application. Removing those increased the
agreement between the Pulse application estimations and sphygmomanometer
measurements to a mean bias of $-4.25$ with 95\% limits of agreement $-13.43$
to $4.93$ bpm.

\section{Future work} \label{sec:conclusions:future}

Having developed a lightweight, real-time \evm{}-based method for the
Android platform which goal is to amplify color variations, the performance
of different variants of the \evm{} method could be improved. Leading
to increase the usage of this method in all kinds of devices.

The implemented Pulse application needs to improve its heart rate estimations
accuracy, and different monitored vital signs could be added, such as,
breathing rate.

To support some features of the implemented Android application, Pulse,
part of the \emph{OpenCV for Android SDK} library had to be modified. These
changes can be contributed back to the community, since the OpenCV support
for the Android platform is still in its early stage.
