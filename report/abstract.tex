\chapter*{Abstract}

\evm{} is a recently presented method capable of revealing temporal
variations in videos that are impossible to see with the naked eye.
Using this method, it is possible to visualize the flow of blood as
it fills the face. From its result, a person's heart rate is possible
to be extracted.

This research work was developed at \emph{Fraunhofer Portugal}
and its goal is to test the feasibility of the
implementation of the \evm{} method on smartphones by developing
an \emph{Android} application for monitoring vital signs based on
the \evm{} method.

There has been some successful effort on the assessment of vital
signs, such as, heart rate, and breathing rate, in a contact-free
way using a webcamera and even a smartphone. However, since the
\evm{} method was recently proposed, its implementation has not
been tested in smartphones yet.Thus, the \evm{} method performance
for color amplification was optimized in order to execute on an Android
device at a reasonable speed.

The Android application implemented includes features, such as,
detection of a person's cardiac pulse, dealing with artifacts' motion,
and real-time display of the magnified blood flow.
Then, the application measurements were evaluated through tests
with several individuals and compared to the ones detected by
the \emph{ViTrox} application and to the readings of a sphygmomanometer.


% \chapter*{Resumo}

% O Resumo fornece ao leitor um sumário do conteúdo da dissertação.
% Deverá ser breve mas conter detalhe suficiente e, uma vez que é a porta
% de entrada para a dissertação, deverá dar ao leitor uma boa impressão
% inicial.

% Este texto inicial da dissertação é escrito no fim e resume numa
% página, sem referências externas, o tema e o contexto do trabalho, a
% motivação e os objectivos, as metodologias e técnicas empregues, os
% principais resultados alcançados e as conclusões.

% Este documento ilustra o formato a usar em dissertações na \Feup.
% São dados exemplos de margens, cabeçalhos, títulos, paginação, estilos
% de índices, etc.
% São ainda dados exemplos de formatação de citações, figuras e tabelas,
% equações, referências cruzadas, lista de referências e índices.

% Seguem-se umas notas breves mas muito importantes sobre a versão
% provisória e a versão final do documento.
% A versão provisória, depois de verificada pelo orientador e de
% corrigida em contexto pelo autor, deve ser publicada na página
% pessoal de cada estudante/dissertação, juntamente com os dois
% resumos, em português e em inglês; deve manter a marca da água,
% assim como a numeração de linhas conforme aqui se demonstra.

% A versão definitiva, a produzir somente após a defesa, em versão
% impressa (dois exemplares com capas próprias FEUP) e em versão
% eletrónica (6 CDs com "rodela" própria FEUP), deve ser limpa da marca de
% água e da numeração de linhas e deve conter a identificação, na primeira
% página, dos elementos do júri respetivo.
% Deve ainda, se for o caso, ser corrigida de acordo com as instruções
% recebidas dos elementos júri.
